\section{Trusted Computing}

Trusted Computing (TC) is the notion of embedding security into the hardware,
making it readily available on commodity devices.
The notion originated in the Trusted Computing Group\footnote{\url{https://trustedcomputinggroup.org}}.
The original goal of TC was the development of Trusted Platform Modules (hardware/firmware)
and more recently Trusted Network Connect (TNC) a protocol specification for secure network connections
with authentication, authorization and accountability.

Some general features present in TC are:
\begin{itemize}
    \item \textbf{Secure Boot} - which allows the system to boot into a defined and trusted configuration.
    \item \textbf{Curtained Memory} - provides strong memory isolation (i.e. memory that cannot be read by other processes).
    \item \textbf{Sealed Storage} - Allows software to keep cryptographically secure secrets.
    \item \textbf{Secure I/O} - Thwarts attacks like keyloggers and screen scrapers.
    \item \textbf{Integrity Measurements} - Ability to compute hashes of executable code,
    configuration data, and other system state information.
    \item \textbf{Remote Attestation} - Allows a trusted device to present reliable evidence to remote parties of the about the running software.
\end{itemize}

\paragraph{Software Attestation}
Software attestation can be divided into local and remote attestation.
Local attestation is the ability of a program to authenticate itself.
Remote attestation is the ability of a system to make reliable statements about the software it is running in another system.

\subsection{Trusted Platform Module}
A Trusted Platform Module (TPM) is hardware designed according to the ISO/IEC 11889 standard.
It is a chip built into a dedicated microcontroller which can be used as a secure cryptographic processor.

\subsubsection{Limitations}
\begin{itemize}
    \item Cross-device data sharing/communication. No secure key exchange protocols as a bootstrapping process for secure interactions, no protected X509 authentication and management, no relevant cryptography.
    \item Trusted clocks, time-based attacks.
    TPMs combine a trusted timer with a secure, non volatile counter,
    for every tick received the counter is increased,
    every $n$ increments the value is stored in NV storage.
    \item No monotonicity. The TPM is not able to keep track of time when not running.
    Whenever the system is booted, the value in the NV memory is picked up, violating monotonicity.
    \item NV Constraints. The NVRAM is small ($1\sim1.5$KB size).
    \item Cryptographic limitations.
    \item Performance issues. The chips are slow.
\end{itemize}

\subsubsection{Issues}
The has been some resistance and criticism over the generalized adoption of TPMs such as:
\begin{itemize}
    \item Rise of privacy concerns.
    \item Physical/Administrative access to hardware bases.
    \item The private endorsement key is fundamental to the security of the TPM circuit,
    this key is never made available to the end-user, requiring users to trust the manufacturer.
    \item Rigidity and lack of relevant trusted functions for specific deployments.
    \item The performance of TPMs is lackluster.
    Vendors have little incentive to use faster (and more expensive) parts.
\end{itemize}

\subsubsection{Services}
TPMs provide three basic services, authenticated boot, attestation and encryption facilities.

\paragraph{Authentication}
Provides mean to unequivocally identify the TPM module and platform.
Each TPM has a embedded key pair, the \textit{Endorsement Key}.
This key is signed by the manufacturer as to guarantee the correctness the chip and validity of the key.

\paragraph{Trusted Boot}
The function responsible for booting the OS in stages,
ensuring that each portion of the OS is trusted and approved for use.
A TC-enabled system using TPM devices maintains a list of appropriate HW and SW components for verification of the boot process.

\paragraph{Authenticated Boot}
Responsible for booting the entire OS in successive stages,
assuming that each portion of the OS loaded is a version that is approved for use.
At each stage the digital signature associated with the software is verified and loaded and
the tamper evidence log of the entire boot process is registered to detect any tampering with the log.

\paragraph{Platform Configuration Registers}
PCRs are special registers which are primarily used to store fingerprints of a portion of the software booting on a computer,
they are guaranteed to be reset upon a computer boot.

\paragraph{Extended Authenticated Configurations}
From the base functions it is possible to expand the trust-boundary to include new hardware devices,
software utilities, stack layers or applications.

\paragraph{Certification Service}
Once a configuration is achieved and logged the TPM can certify the configuration to others, producing a digital certificate.
There is confidence that the configuration is unaltered because the TPM is considered to be trustworthy and only the TPM possesses the TPM private key.

\paragraph{Remote Attestation Protocol}
Aims to provide a remote entity with means to assess the security if the contacted platform.
Allows to retrieve from the TPM a signature called \textit{quote} that tells the resulting platform configuration,
built through the attested platform boot, by the loaded firmware and software.

\paragraph{Encryption Service}
The TPM keeps a secret master key, unique to the machine.
\begin{itemize}
    \item \textbf{Binding Encryption} encrypts data using the TPM binding key.
    \item \textbf{Sealing Encryption} encrypts data so that it can only be decrypted by a machine with a certain verified configuration.
\end{itemize}

\subsubsection{Functions}
\paragraph{I/O}
All commands enter/exit through the I/O bus, it is the only way to communicate with the TPM functions.
\paragraph{Cryptography Co-Processor}
Processor for encryption functions, it provides utilities such as hashing, RSA encryption and digital signatures
as well as AES encryption.
\paragraph{Key Generation}
Able to create RSA keypairs and generate AES symmetric keys.
\paragraph{HMAC \& SHA-1 Engines}
Provides hardware HMAC and SHA-1 implementations.
\paragraph{Random Number Generator}
Useful for key-generation, nonces and generally as a source of randomness.
\paragraph{Power Detection}
Manages the TPM power state in conjunction with the platform power state.
\paragraph{Execution Engine}
Runs program code to dispatch the received commands.
\paragraph{Non-Volatile Memory}
Used to store TPM parameters in a persistent manner.
\paragraph{Volatile Memory}
Temporary storage for the execution engine, as well as for volatile parameters.

\subsection{Trustworthy Computing}
Trustworthy Computing is about systems that will not compromise security,
in contrast to TC in which we can just expect the system to be safe.
\paragraph{System Integrity Guarantees}
Hardware verification and software components verification, such as what the components are and what data can they access.
\paragraph{Data Security Environment}
Authentication of the machine unique identity, crypto keys stored in sealed storage and all critical data.
\paragraph{Privacy Protection}
Prevention of unauthorized access from the network, user control policy improvements and domain separation for data leakage avoidance.

\subsection{Trusted Execution Environment}
Trusted Execution Environment (TEE) provide isolated execution with memory isolation and runtime integrity guarantees.
They also provide secure I/O, secure storage,
more flexibility since they act as TPMs but directed to the application-level.
Furthermore they provide virtualization security, measurement, attestation and sealing.

\subsubsection{Hardware-Backed TEEs}
The main idea behind it is to make use of the secure area of a processor to support the TEE.
It guarantees that code and data loaded inside to be protected with respect to confidentiality and integrity.
The TEE provides security features such as:
\begin{itemize}
    \item Isolated Execution.
    \item Integrity of Trusted Applications along with confidentiality of their assets.
\end{itemize}

\subsection{Intel SGX}
Intel SGX is a technology developed by Intel which provides secure computation features, such as memory isolation through enclaves.
Protection from software attacks is provided by enclaves, which works was follows.
\begin{enumerate}
    \item App is build with trusted and untrusted parts.
    \item App runs and creates an enclave which is placed in trusted memory.
    \item Trusted function is called,
    code running inside the enclave sees data in clear and
    external access to the data is denied.
    \item The function returns and the enclave data remains in trusted memory.
\end{enumerate}

An enclave:
\begin{itemize}
    \item Has its own code and data.
    \item Provides confidentiality and integrity.
    \item Has controlled entry points.
    \item Supports multiple threads and has full access to the application memory.
\end{itemize}

\subsubsection{SW Attestation with Enclaves}
\begin{enumerate}
    \item The enclave is built and measured.
    \item The enclave requests \texttt{REPORT} (an hardware signed blob that includes the enclave identity information).
    \item The signed request \texttt{REPORT} is sent to the server.
    \item \texttt{REPORT} is verified.
    \item The application key is sent to the enclave.
    \item Enclave-Platform Specific Sealing Key is generated.
    \item Application key encrypted using the Sealing Key and stored for later use.
\end{enumerate}