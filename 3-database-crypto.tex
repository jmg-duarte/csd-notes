\section{Database Cryptography}
\subsection{Homomorphic Encryption}

Homomorphic encryption is a form of encryption that allows operations over the encrypted results
as if the operations were done on the original data.

Homomorphic encryption can be categorized over three different methods.

\subsubsection{Encryption Methods}

\paragraph{Fully Homomorphic Encryption}
It is considered to be the holy grail of cryptography and while it has been proven to be theoretically possible there has been no progress yet.
FHE supports unlimited additions and multiplications.

\paragraph{Partially Homomorphic Encryption}
PHE supports only one method, either additions or multiplications.

\paragraph{Somewhat Homomorphic Encryption}
SWHE tries to bridge the gap between methods providing both in a limited manner.

\subsubsection{Beyond HE Methods}
While HE provides versatility to the processing of encrypted data it is not adequate for every purpose.
Some factors are required to be considered such as.
\begin{itemize}
    \item The output is encrypted,
    in contrast to other existing obfuscation and functional encryption techniques,
    which retrieve the result as plaintext.
    \item All inputs must be encrypted with the same key,
    that is, the key that encrypts the original data is also required to encrypted the operation input.
    \item HE does not provide authentication or integrity guarantees on computations,
    the assumption is that the algorithms are correct and results are obtained by the correct execution.
\end{itemize}

\subsection{CryptDB}
CryptDB tries to address information leakage by encrypting the DB content.
It provides practical confidentiality and encrypted query processing.

\subsubsection{Adversary Model}
\paragraph{Honest-but-Curious Admin}
The attacker may be the DB admin trying to extract information over client documents.
This actor is considered to have full control over the DBMS, even access to RAM.
CryptDB aims to provide confidentiality, the attacker is assumed to be passive,
only trying to extract information, not modifying information.

\paragraph{Leaked Keys}
The adversary may have gotten access to the keys used to encrypt the entire data base.
An adversary of who compromises the application server or the proxy can only decrypt data of currently logged-in users.
The countermeasure CryptDB provides is not letting the inactive user keys to be available to the attacker.

\subsubsection{Proposed Techniques}
\paragraph{Using PHE}
This provides a set of primitive operators such as equality checks, order comparisons and aggregations.
It also provides a new construction for joins.

\paragraph{Adjustable Query-Based Encryption}
Use onion encryption, that is, encrypt using several methods one after the other.

\paragraph{Chain-Encryption Keys}
Onion encryption chains, based on the password of one of the users with data access authorization.
Guarantee that data is inaccessible if authorized users are not logged in.
Capture application based-access control requirements and support data-sharing by a delegation model.

\subsection{Cipherbase}
Cipherbase was introduced by Microsoft,
it is a full-fledged SQL database providing high performance and high data confidentiality.
It support stored procedures, indexes, recovery procedures as well as full SQL,
column-level encryption and other encryption schemes.
Cipherbase makes use of special hardware for encryption processes.