\section{Database Security}

\subsection{Database Architecture}
A Database Management System (DBMS) is composed of several elements such as:
\begin{itemize}
    \item DB Management Functions
          \begin{itemize}
              \item User Management
              \item DB Utilities (Administration and Maintenance)
              \item External Query Applications
          \end{itemize}
    \item DB Languages (such as SQL and DDL)
    \item Persistency and OLTP services
          \begin{itemize}
              \item Transaction Management \& Concurrency
              \item File Management
              \item Authorization Access Tables
              \item Physical DB Mapping
          \end{itemize}
\end{itemize}

\subsubsection{Relational DBMS}
These are built upon now common elements such as:

\begin{itemize}
    \item Entity models and relations
    \item Tables - and their elements such as rows and columns
    \item Keys - primary and foreign keys
    \item Views
    \item SQL
    \item User Defined Functions
\end{itemize}

\paragraph{SQL}
Structure Query Language (SQL) is the language databases provide for query operations,
each database provides its own "flavor" of SQL.

\paragraph{SQL Injection Attacks}
SQL injection attacks (SQLi) are one of the most common and dangerous threats,
the intuition is to send a string which can be interpreted as an SQL command,
they can be used to retrieve information, manipulate data or even launch a denial of service attack.
When it comes to preventing SQLi attacks, there are several countermeasures available,
such as:
\begin{itemize}
    \item Defensive Coding
          \begin{itemize}
              \item Data Sanitization and Validation
              \item Static-Analysis (Application Wise)
          \end{itemize}
    \item Intrusion Detection
          \begin{itemize}
              \item Signature-based
              \item Anomaly-based
              \item Code analysis
          \end{itemize}
    \item Runtime Prevention
          \begin{itemize}
              \item Runtime filters with dynamic detection
              \item Sensors in different nodes of routing SQL statements
              \item Effects on DB state
          \end{itemize}
\end{itemize}

\subsection{Access Control}
Techniques used to grant and revoke authorizations.

\subsubsection{Cascading Authorizations}
Granting/revoking authorizations will cascade throughout the chain of permissions.
That is, if a users authorization $A$ is revoked,
authorizations given by means of $A$ will also be revoked.
This mechanism can also be time based.

\subsubsection{DAC vs. RBAC}
\paragraph{Discretionary Access Control}
Typically applies permissions in a "per-\texttt{userid}" base,
or organize users according to categories such as the end users, application owners, DB admins, etc.

\paragraph{Role-Based Access Control}
The intuition behind RBAC is to extend the DAC model by defining more "fine-grain" roles.
Allowing roles to be specific for role creation/deletion, permission assignment and more.
This approach also allows to establish permissions between roles.

\subsubsection{Typical Access Control Management Policies}
Several common AC administration policies exist and work for different goals.

\paragraph{Centralized}
A small number of privileged users to grant and revoke rights.

\paragraph{Ownership-based}
The owner of a table may grant and revoke rights to the table.

\paragraph{Decentralized}
In addition to granting and revoking access rights to a table,
the owner of the table may grant/revoke authorization rights to other users,
allowing them to grant/revoke access rights to the table.

\subsection{Inferential Threats}
Inference can be considered to be indirect access to information.

\paragraph{Inferential Attacks}
Inferential attacks are usually performed by insider attackers,
or by intruders who obtained valid privileges.
An inference action is the process of performing authorized queries and deducing unauthorized information from legitimate responses from the server.

\paragraph{Inference Vulnerability}
An inference vulnerability exists when the combination of a number of data items is more sensitive that the individual items and their specific access authorization.

\paragraph{Inferential Metadata}
Inferential metadata refers to the knowledge of the DB structural model,
as well as correlations or dependencies among data items.

\paragraph{Inferential Channels}
The inference channel is defined as the information transfer path by which unauthorized data is accessed by a sequence and combination of authorized data.

\subsection{Database Encryption}
Database encryption is the last line of defense against malicious actors.
There exist several challenges related to DB encryption.

\subsubsection{Key Management}
Authorized users must access keys used to encrypt and decrypt the stored data.
However it is also necessary to support a wide range of users and external applications.
Furthermore it is necessary to manage several keys.

\subsubsection{Operation Support}
The database is required to support SQL as if it was not encrypted,
this is not a simple problem as it conflicts with one of the main goals of cryptography,
which is to make plain data imperceptible.

\subsubsection{Scope of Encryption}
DB encryption can be applied to many levels from the entire database to the field as a single unit.
The finer the encryption, the more keys are required to manage.

\subsubsection{DB Encryption Solution}
Encrypted DB version (tables) built and released by the data owner entity as a set of rows,
with contiguous binary blocks.

\begin{equation*}
    \begin{split}
        B_i & = (X_{i1} \parallel X_{i2} \parallel \dots \parallel X_{iM}) \\
        E(k, B_i) & = E(k, (X_{i1} \parallel X_{i2} \parallel \dots \parallel X_{iM}))
    \end{split}
\end{equation*}

The client now must encrypt the data for retrieval.
In the client front-end proxy indexes are associated with each table and for some or all of the attributes an index value is created.
So we have that for each row of the unencrypted DB the mapping is:
\begin{equation*}
    (X_{i1}, X_{i2}, \dots, X_{iM}) \Rightarrow [E(k, B_i), I_{i1}, I_{i2}, \dots, I_{iM}]
\end{equation*}

\begin{equation*}
    \begin{matrix}
        E(k, B_1) & I_{11} & I_{12} & \dots  & I_{1M} \\
        E(k, B_2) & I_{21} & I_{22} & \dots  & I_{1M} \\
        \vdots    & \vdots & \vdots & \ddots & \vdots \\
        E(k, B_N) & I_{N1} & I_{N2} & \dots  & I_{NM} \\
    \end{matrix}
\end{equation*}

To create the indexes $I_{ij}$ we generate ranges over the attribute values (covering all possible values) without overlapping partitions.